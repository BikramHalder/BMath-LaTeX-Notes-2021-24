\chapter*{Lecture 5, February 4}
\addcontentsline{toc}{chapter}{Lecture 5, Febuary 4}
\setcounter{chapter}{5}
\setcounter{section}{0}

Let $f \in \mathcal{B}[a,b]$. We have already shown that for any partition $P \in \mathcal{P}[a,b]$, we have 
\begin{equation}\label{eq:LSU}
    L(f,P) \leq S(f,P) \leq U(f,P) 
\end{equation}

\section{Riemann Sums as a tool for computing integrals}

\begin{defn}
    Given $f \in \mathcal{B}[a,b]$, we say that 
    \begin{equation}\label{eq1:feb4}
        \lim_{\|P\|\to 0} S(f,P) = \lambda 
    \end{equation}
    for some $\lambda \in \bb{R}$, if for every $\eps > 0$, there exists a $\delta > 0$ such that 
    \[ |S(f,P) - \lambda| < \eps \]
    for all $P \in \mathcal{P}[a,b]$, and tag set $T_P$, such that $\|P\| < \delta$.
\end{defn}

Observe that whenever the limit in equation $(\ref{eq1:feb4})$ exists, the limit is unique, and hence our definition does not have any ambiguity. 

\begin{thm}\label{thm1:feb4}
    Let $f \in \mathcal{B}[a,b]$. Then $f \in \mathcal{R}[a,b]$, if and only if there exists a $\lambda \in \bb{R}$ such that \[ \lim_{\|P\| \to 0} S(f,P) = \lambda\]
    in this case we have $\int_a^b f = \lambda$.
\end{thm}
\begin{prf}
    Let $f \in \bb{R}[a,b]$, and suppose $\lambda = \int_a^b f$. Fix $\eps > 0$. Then from \texty{Darboux Criterion} we get that there exists a $\delta > 0$ such that 
    \[ U(f,P) - L(f,P) < \eps \]
    for all $P \in \mathcal{P}[a,b]$ such that $\|P\| < \delta$. Then note that 
    \begin{align*}
        L(f,P) > U(f,P) - \eps \geq \overline{\int_a^b}f - \eps = \lambda - \eps \\ 
        U(f,P) < L(f,P) + \eps \geq \underline{\int_a^b}f + \eps = \lambda + \eps
    \end{align*}
    Then from equation $(\ref{eq:LSU})$ we get that 
    \[ \lambda - \eps < L(f,P) \leq S(f,P) \leq U(f,P) < \lambda + \eps \]
    Hence, we get that $|S(f,P) - \lambda| < \eps$ for all $P \in \mathcal{P}[a,b]$ and tag set $T_P$, such that $\|P\| < \delta$.

    Conversely, suppose \[ \lim_{\|P\| \to 0} S(f,P) = \lambda \]
    Then for $\eps > 0$, we get there exists a $\delta > 0$ such that $|S(f,P) - \lambda| < \displaystyle{\frac{\eps}{2}}$ for all $P \in \mathcal{P}[a,b]$ and tag set $T_P$, such that $\|P\| < \delta$. Suppose $P = \{x_0, \dots, x_n\}$, and let $M_j = \sup_{x \in I_j} f(x)$ where $I_j = [x_{j-1},x_j]$ for $j = 1, \dots, n$. Then note that from properties of supremum of a set we get that $\forall \ \eps > 0$, there exists a $\zeta_j \in I_j$ such that \[ M_j - \frac{\eps}{2(b-a)} < f(\zeta_j) \leq M_j \] So if we choose our tag set to be the set of all these $\zeta_j$'s, we get that 
    \begin{align*}
        S(f,P) &= \sum_{j=1}^n f(\zeta_j) \Delta x_j \\ 
               &> \sum_{j=1}^n \left(M_j - \frac{\eps}{2(b-a)}\right) \Delta x_j \\ 
               &= \sum_{j=1}^n M_j \Delta x_j - \frac{\eps}{2(b-a)} \sum_{j=1}^n \Delta x_j \\ 
               &= U(f,P) - \frac{\eps}{2}
    \end{align*}
    But now since 
    \[
        \lambda-\frac{\eps}{2} < S(f,P) < \lambda + \frac{\eps}{2}  
    \]
    for all $P \in \mathcal{P}[a,b]$ and tag set $T_P$, such that $\|P\| < \delta$, we get that 
    \[
        U(f,P) - \frac{\eps}{2} < S(f,P) < \lambda + \frac{\eps}{2} \Rightarrow U(f,P) < \lambda + \eps   
    \]
    and 
    \[
        \lambda - \frac{\eps}{2} < S(f,P) \leq U(f,P) \Rightarrow U(f,P) > \lambda - \eps   
    \]
    hence, combining everything we get that 
    \begin{equation}\label{eq3:feb4}
        |U(f,P) - \lambda| < \eps, \ \forall \, P \in \mathcal{P}[a,b] \mbox{ such that } \|P\| < \delta
    \end{equation}
    Now using similar arguments its easy to show that 
    \begin{equation}\label{eq4:feb4}
        |L(f,P) - \lambda| < \eps, \ \forall \, P \in \mathcal{P}[a,b] \mbox{ such that } \|P\| < \delta
    \end{equation}
    Thus from equation $(\ref{eq3:feb4})$ and $(\ref{eq4:feb4})$ we get that 
    \[
        U(f,P) - L(f,P) \leq |U(f,P) - \lambda| + |L(f,P)-\lambda| < 2\eps  
    \] 
    for all $P \in \mathcal{P}[a,b]$ such that $\|P\| < \delta$, hence from \texty{Darboux's criterion} it follows that $f \in \mathcal{R}[a,b]$, and finally since 
    \[ \lambda - \eps < L(f,P) \leq \int_a^b f \leq U(f,P) < \lambda + \eps \]
    It follows that $\int_a^b f = \lambda$, which completes the proof.
\end{prf}

\ 

\texty{Theorem} $\ref{thm1:feb4}$ is a nice tool for computing integrals, if we already know that $f \in \mathcal{R}[a,b]$.

\begin{thm}\label{thm2:feb4}
    Suppose $f \in \mathcal{R}[a,b]$ and $\{P_n\}_{n\in\bb{N}} \subset \mathcal{P}[a,b]$ such that $\| P_n \| \to 0$ as $n \to \infty$. Then 
    \[ 
        \lim_{n \to \infty} S(f,P_n) = \int_a^b f    
    \]
\end{thm}
\begin{prf}
    Since $f \in \mathcal{R}[a,b]$, using \texty{Darboux criterion} we get that for all $\eps > 0$, there exists a $\delta > 0$, such that 
    \[
        U(f,P) - L(f,P) < \eps, \ \forall \, P \in \mathcal{P}[a,b] \mbox{ such that } \|P\| < \delta  
    \]
    Now since $\|P_n\| \to 0$, thus for all $\delta > 0$, there exists an $N \in \bb{N}$ such that $\|P_n\| < \delta, \ \forall \, n > N$. Thus, we get 
    \[ 
        U(f, P_n) - L(f, P_n) < \eps, \ \forall \, n > N    
    \]
    But then we get that 
    \[ 
        \left( U(f,P_n) - \int_a^b f \right) + \left( \int_a^b f - L(f,P_n) \right) < \eps, \ \forall \, n > N    
    \]
    Note that $U(f,P_n) - \int_a^b f \geq 0$ and $\int_a^b f - L(f,P_n) \geq 0$, hence we have 
    \begin{align}\label{eq5:feb4}
        &&0 \leq U(f,P_n)-\int_a^b f < \eps &&\mbox{and} &&0 \leq \int_a^b f - L(f,P_n) < \eps &&\forall \, n > N
    \end{align}
    Thus from equation $(\ref{eq5:feb4})$, we conclude that 
    \[
        \lim_{n\to\infty} L(f,P_n) = \lim_{n\to\infty} U(f,P_n) = \int_a^b f  
    \]
    Finally from \texty{Squeeze theorem} we get that 
    \[
        \int_a^b f = \lim_{n\to\infty} L(f,P_n) \leq \lim_{n\to\infty}S(f,P_n) \leq \lim_{n\to\infty} U(f,P_n) = \int_a^b f  
    \]
    and hence $\lim_{n\to\infty} S(f,P_n) = \int_a^b f$, which completes the proof.
\end{prf}

\begin{exampleBox}
    \texty{Note:} \textit{The limit $\lim_{n\to\infty} S(f,P_n) = \int_a^b f$ does not depend on the tag set, so we can choose our favourite tag set for computing the integral, given that we know that the function is Riemann integrable.}
\end{exampleBox}

Suppose we know that $f \in \mathcal{R}[a,b]$, then consider the following partition \[ a = x_0 < x_1 = x_0 + \frac{1}{n}(b-a) < x_2 = x_0 + \frac{2}{n}(b-a) < \cdots < x_n = x_0 + (b-a) = b \]
Then note that if $P = \{x_0,\dots,x_n\}$, then $\|P\| = \frac{b-a}{n}$, thus we get that $\|P\| \to 0$ as $n \to \infty$. And since $f \in \mathcal{R}[a,b]$ we know that 
\begin{align*}
    \lim_{n\to\infty} S(f,P_n) = \int_a^b f
\end{align*}
Now we choose the tag set to be the points $\{x_0,x_1,\dots,x_{n-1}\}$, then we get that 
\begin{equation}\label{eq6:feb4}
    \lim_{n\to\infty}\left(\frac{b-a}{n}\sum_{j=1}^n f \left(a+\frac{j-1}{n}(b-a) \right)\right) = \int_a^b f
\end{equation}
Thus, if we know that the function is Riemann integrable then the integral can be written as a limit of \texty{Newton sums}.

\section{What else can we say about the set $\mathcal{R}[a,b]$ ?}

Observe that the set of bounded functions on $[a,b]$, i.e., $\mathcal{B}[a,b]$ forms a vector space over the field of real numbers, with addition on $\mathcal{B}[a,b]$ as sum of functions, and scalar multiplication as product of a function with a real number. 
\begin{itemize}
    \item Associativity and commutativity follows trivially.
    \item Additive identity is the zero function, and obviously we have $\mathbf{0} \in \mathcal{B}[a,b]$, and multiplicative identity is $1$.
    \item And if $f \in \mathcal{B}[a,b]$, then since $|f| = |-f|$, we get that $-f \in \mathcal{B}[a,b]$, and since $f + (-f) = \mathbf{0}$, we get that additive inverse exists for all $f \in \mathcal{B}[a,b]$.
    \item Distributive properties hold trivially.
\end{itemize} 
Also, it's not difficult to show that if $f : [a,b] \to [c,d]$ and $g : [c,d] \to \bb{R}$ such that both $f,g \in \mathcal{B}[a,b]$, then $g \circ f : [a,b] \to \bb{R} \in \mathcal{B}[a,b]$.

The next question that arises immediately is can we say $\mathcal{R}[a,b]$ is a vector space over the field of real numbers? Well, with similar operations as in case of $\mathcal{B}[a,b]$, again it's not very difficult to show that indeed $\mathcal{R}[a,b]$ is a vector space over the field of real numbers. Now consider the function $\mathcal{I} : \mathcal{R}[a,b] \to \bb{R}$ given by 
\begin{equation}\label{eq7:feb4}
    \mathcal{I}(f) = \int_a^b f, \ \ \forall \, f \in \mathcal{R}[a,b]
\end{equation}

\begin{thm}
    Suppose $\mathcal{I} : \mathcal{R}[a,b] \to \bb{R}$ be defined as in equation $(\ref{eq7:feb4})$, then following conditions are true 
    \begin{enumerate}
        \item[(i)] For all $r,s \in \bb{R}$ and $f,g \in \mathcal{R}[a,b]$ we have 
        \[
            \mathcal{I}(rf+sg) = r\mathcal{I}(f) + s\mathcal{I}(g)
        \]  
        i.e., $\mathcal{I}$ is a \texty{linear functional}.
        
        \item[(ii)] $\mathcal{I}$ preserves the order, i.e., if $f, g \in \mathcal{R}[a,b]$ and we have $f(x) \leq g(x), \ \forall \, x \in [a,b]$, then \[ \mathcal{I}(f) \leq \mathcal{I}(g) \]
        
        \item[(iii)] Let $c \in (a,b)$ then we have 
        \[
            \mathcal{I}(f) = \int_a^c f + \int_c^b f  
        \]  
    \end{enumerate}
\end{thm}

\begin{prf}
\begin{enumerate}
    \item[(i)] Let $f,g \in \mathcal{R}[a,b]$, we will first show that $f+g \in \mathcal{R}$. Note the since $f,g \in \mathcal{R}$, then using \texty{theorem} $\ref{thm1:feb4}$ we get for $\eps > 0$, there exists $\delta_1 > 0$ and $\delta_2 > 0$, such that 
    \begin{align*}
        &\left|S(f,P) - \int_a^b f \right| < \frac{\eps}{2}, \ \forall \, P \in \mathcal{P}[a,b] \mbox{ such that } \|P\| < \delta_1 \\ &\left|S(g,P) - \int_a^b g\right| < \frac{\eps}{2}, \ \forall \, P \in \mathcal{P}[a,b] \mbox{ such that } \|P\| < \delta_2 
    \end{align*}
    choose $\delta = \min \{\delta_1,\delta_2\}$, then we get that 

    \begin{align*}
        \left| S(f+g, P) - \int_a^b f - \int_a^b g \right| &= \left| \left(S(f,P) - \int_a^b f \right) + \left( S(g,P) - \int_a^b g \right) \right| \\ 
        &\leq \left| S(f,P) - \int_a^b f \right| + \left| S(g,P) - \int_a^b g \right| \\ 
        &\overset{(1)}{<} \eps 
    \end{align*}
    where the $(1)$ holds $\forall \, P \in \mathcal{P}[a,b] \mbox{ such that } \|P\| < \delta$. Hence, we have shown that $f+g \in \mathcal{R}[a,b]$, and in particular we have 
    \begin{equation}\label{eq9:feb4}
        \int_a^b (f+g) = \int_a^b f + \int_a^b g 
    \end{equation}
    Finally its enough to show that if $f \in \mathcal{R}[a,b]$, then $rf \in \mathcal{R}[a,b]$ for all $r \in \bb{R}$. But this is not difficult to prove as it can be easily seen that 
    \[
        S(rf,P) = rS(f,P)    
    \]
    and hence 
    \begin{align*}
        \left| S(rf,P) - r \int_a^b f \right| = |r|\left| S(f,P) - \int_a^b f \right| \overset{(1)}{<} |r|\eps
    \end{align*}
    where $(1)$ holds for all $P \in \mathcal{P}[a,b]$, such that $\|P\| < \delta$, from here we can conclude that $rf \in \mathcal{R}[a,b]$, and in particular we have shown that
    \begin{equation}\label{eq10:feb4}
        \int_a^b rf = r \int_a^b f  
    \end{equation} 
    Hence, using equation $(\ref{eq9:feb4})$ and $(\ref{eq10:feb4})$ we get that if $f,g \in \mathcal{R}[a,b]$, then $rf,sg \in \mathcal{R}[a,b]$, and hence $rf+sg \in \mathcal{R}[a,b]$, and in particular we have 
    \[
        \mathcal{I}(rf+sg) = \int_a^b (rf+sg) = r\int_a^b f + s \int_a^b g = r\mathcal{I}(f) + s\mathcal{I}(s)
    \]
    Thus we have proved that $I : \mathcal{R}[a,b] \to \bb{R}$, is a \texty{linear functional}.

    \item[(ii)] Let $f,g \in \mathcal{R}[a,b]$ such that $f(x) \leq g(x), \ \forall \, x \in [a,b]$, then consider a sequence of partitions $\{P_n\} \subset \mathcal{P}[a,b]$ such that $\|P_n\| \to 0$ as $n \to \infty$, then using \texty{theorem} $\ref{thm2:feb4}$, we get that 
    \begin{align*}
        &&\lim_{n\to\infty} S(f,P_n) = \mathcal{I}(f) &&\mbox{and} &&\lim_{n\to\infty} S(g,P_n) = \mathcal{I}(g)    
    \end{align*} 
    then for any $P_n \in \mathcal{P}[a,b]$ and for any tag set $T_{P_n} = \{\zeta_1, \dots, \zeta_n\}$ we get that 
    \[ 
        S(f,P_n) = \sum_{j=1}^n f(\zeta_j) \Delta x_j \leq \sum_{j=1}^n g(\zeta_j) \Delta x_j = S(g,P_n)    
    \] 
    and hence we get that 
    \[
        \mathcal{I}(f) = \lim_{n\to\infty} S(f,P_n) \leq \lim_{n\to\infty} S(g,P_n) = \mathcal{I}(g)  
    \] 

    \item[(iii)] Let $c \in (a,b)$. We first need to show that if $f \in \mathcal{R}[a,b]$, then $f \in \mathcal{R}[a,c]$ and $f \in \mathcal{R}[c,b]$ for all $c \in [a,b]$. Now since $f \in \mathcal{R}[a,b]$ \texty{Cauchy criterion} tells us that for all $\eps > 0$, there exists a partition $P_{\eps} \in \mathcal{P}[a,b]$ such that for all $P \supset P_{\eps}$, we have 
    \[
        U(f,P) - L(f,P) < \eps    
    \] 
    Now \gen we can assume that $c \in P$ (otherwise we can simply work with $P \cup \{c\}$). Let \[P = \{a = x_0, \dots, x_{i-1}, x_i = c, x_{i+1}, \dots, x_n = b\ \] and consider $P_1 = \{x_0,\dots,x_i\}$ and $P_2 = \{x_i,\dots,x_n\}$, then we have $P_1 \in \mathcal{P}[a,c]$ and $P_2 \in \mathcal{P}[c,b]$ and further $P = P_1 \cup P_2 $. Now observe that 
    \begin{align*}
        &&L(f,P) = L(f,P_1) + L(f,P_2) &&\mbox{and} &&U(f,P) = U(f,P_1) + U(f,P_2)
    \end{align*}
    Then we get that 
    \begin{equation}\label{eq11:feb4}
        \underbrace{(U(f,P_1) - L(f,P_1))}_{\geq 0} + \underbrace{(U(f,P_2) - L(f,P_2))}_{\geq 0} = U(f,P) - L(f,P) < \eps
    \end{equation}
    Thus from equation $(\ref{eq11:feb4})$, we deduce that 
    \begin{align}\label{eq12:feb4}
        &&U(f,P_1) - L(f,P_1) < \eps &&\mbox{and} &&U(f,P_2) - L(f,P_2) < \eps
    \end{align}
    Now since equation $(\ref{eq12:feb4})$, will hold for any partition $P', P^{''}$ finer that $P_1, P_2$ respectively we can conclude that $f \in \mathcal{R}[a,c]$ and $f \in \mathcal{R}[c,b]$, so $\int_a^c f $ and $\int_c^b f$ are well-defined. Finally, we have 
    \begin{align*}
        \int_a^b f &\geq L(f,P) \\ 
                   &= L(f,P_1) + L(f,P_2) \\ 
                   &> (U(f,P_1)-\eps) + (U(f,P_2)-\eps) \\ 
                   &= (U(f,P_1)+U(f,P_2)) - 2\eps \\ 
                   &\geq \left(\int_a^c f + \int_c^b f \right) - 2\eps
    \end{align*}
    and on the other hand we get 
    \begin{align*}
        \int_a^b f &\leq U(f,P) \\ 
                   &= U(f,P_1) + U(f,P_2) \\ 
                   &< (L(f,P_1)+L(f,P_2)) + 2\eps \\ 
                   &\leq \left( \int_a^c f + \int_c^b f \right) + 2\eps
    \end{align*}
    Hence, we have shown that for all $\eps > 0$, we have  
    \[
        \left| \int_a^b f - \left( \int_a^c f + \int_c^b f\right) \right| < 2\eps
    \]  
    and hence, we can conclude that 
    \[ 
        \mathcal{I}(f) = \int_a^b f = \int_a^c f + \int_c^b f    
    \]
\end{enumerate}
\end{prf}