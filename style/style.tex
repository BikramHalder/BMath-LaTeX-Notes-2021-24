%This tex file contains all the stylings, so if you render some stylings please leave a comment so that others get aware of it !

\usepackage{tcolorbox}
\usepackage{amsmath}
\usepackage{amssymb}
\usepackage{amsthm}
\usepackage{tcolorbox}
\usepackage[a4paper, total={6in, 9in}]{geometry}
\setlength{\parindent}{0ex}
\usepackage{tikz}
\usepackage{float}
\usepackage{multicol}
\usepackage{caption}
\usepackage[T1]{fontenc}
\usepackage{tgschola}
\usepackage{lmodern}
\usepackage{subcaption}
\usepackage{tocloft}
\usepackage{hyperref}
\usepackage{fancyvrb}
\usepackage{listings} 
\usepackage{color} 
\usepackage{enumitem}
\usepackage{parskip}
\usepackage{bbm}
\usepackage{xcolor}
\usepackage{physics} % For notation used in analysis






\hypersetup{
  colorlinks = true,
  linkcolor = blue,
  urlcolor = cyan
  }
  
% New commands
  
%\renewcommand{\cftsecleader}{\cftdotfill{\cftdotsep}}
\theoremstyle{definition} \newtheorem{lemma}{Lemma}





\renewcommand{\familydefault}{\sfdefault}
\renewcommand{\qed}{\quad $ \blacksquare $}

\newcommand{\contra}{\rightarrow \leftarrow}
\newcommand{\fif}{if and only if }
\newcommand{\gen}{without loss of generality }
\newcommand{\N}{\mathbb{N}}
\newcommand{\Z}{\mathbb{Z}}
\newcommand{\Q}{\mathbb{Q}}
\newcommand{\R}{\mathbb{R}}
\newcommand{\pf}{\mathbb{P}} % pf can stand for either "Probability Function" or "P fancy"
\newcommand{\m}[1]{\pmod {#1}}
\newcommand{\eps}{\epsilon}
\newcommand{\floor}[1]{\lfloor {#1} \rfloor}
\newcommand{\eq}{\equiv}
\newcommand{\msk}{\medskip}
\newcommand{\ssk}{\smallskip}
\newcommand{\bsk}{\bigskip}
\newcommand{\Qed}{\quad \blacksquare }
% \newcommand{\tr}[1]{\operatorname{trace}\left({#1}\right)} % physics already defines this
\newcommand{\modulo}[1]{\ (\mbox{mod } #1)}
\newcommand{\bb}[1]{\mathbb{#1}}


\newcommand{\texty}[1]{\textsf{\textbf{#1}}}
\newcommand{\shared}[1]{\begin{flushright}\textit{(shared by #1)}\end{flushright}}
\newcommand{\plabel}[1]{(Solution to \texty{Problem} \ref{#1})}
\newcommand{\Solun}[1]{\textit{Solution #1}\\}
\newcommand{\rd}[1]{{rd(#1)}}
\newcommand{\X}{\times}
\newcommand{\thatis}{\text{,i.e.,  }}
\newcommand{\Jac}[2]{D\psi^{(#1)}(#2)}
\newcommand{\JacX}[1]{\Bigg|\frac{1}{y}D\psi^{(#1)}(x^{(#1)})\Bigg|}
\newcommand{\Abs}[1]{| #1|}
\newcommand{\AbsB}[1]{\Bigg| #1 \Bigg|}

% Custom tcolorbox styles and new environments

\newtcolorbox{probBox}
{
    colframe = blue!40,
    colback = cyan!4,
    titlerule = 0mm,
    arc = 3mm,
    sharp corners = northwest, 
    sharp corners = southeast,
}

\newtcolorbox{theoremBox}
{
    colback = teal!6,
    colframe = teal!50,
    arc = 0mm,
    leftrule = 2.5mm,
    rightrule = 0mm,
    toprule = 0mm,
    bottomrule = 0mm,
}

\newtcolorbox{lemmaBox}
{
    colback = orange!6,
    colframe = orange!50,
    arc = 0mm,
    leftrule = 2.5mm,
    rightrule = 0mm,
    toprule = 0mm,
    bottomrule = 0mm,
}

\newtcolorbox{corrBox}
{
    colback = magenta!6,
    colframe = magenta!50,
    arc = 4mm,
    sharp corners = southwest,
    sharp corners = northeast,
}

\newtcolorbox{exampleBox}
{
    colback = green!6,
    colframe = green!50,
    arc = 2.5mm
}

\newtcolorbox{clmBox}
{
    colback = magenta!4,
    colframe = magenta!50,
    arc = 5mm,
    leftrule = 0mm,
    rightrule = 2.5mm,
    toprule = 0mm,
    bottomrule = 0mm,
    sharp corners = northeast,
    sharp corners = southeast
}

\newtcolorbox{defBox}
{
  colback = red!4,
  colframe = red!50,
  arc = 0mm,
  leftrule = 2.5mm,
  rightrule = 0mm,
  toprule = 0mm,
  bottomrule = 0mm
}

\newenvironment{soln}
{
    \vspace{0.1cm}
    \textit{Solution.}
}{$\hfill \blacksquare$}

\newenvironment{soln-container}
{
    \vspace{0.1cm}
}

\newtheorem{theorem}{Theorem}[section]
\newtheorem{corollary}{Corollary}[theorem]
% \newtheorem{lemma}[theorem]{Lemma}
\newtheorem{clm}[theorem]{Claim}

\theoremstyle{remark}
\newtheorem{examples}{Example}[theorem]

\theoremstyle{definition}
\newtheorem{prob}{Problem}
\newtheorem{definition}{Definition}[section]

\newenvironment{thm}{\begin{theoremBox} \begin{theorem}}{\end{theorem} \end{theoremBox}}
\newenvironment{problem}{\begin{probBox} \begin{prob}}{\end{prob} \end{probBox}}
\newenvironment{lem}{\begin{lemmaBox} \begin{lemma}}{ \end{lemma} \end{lemmaBox}}
\newenvironment{corr}{\begin{corrBox} \begin{corollary}}{ \end{corollary} \end{corrBox}}
\newenvironment{example}{\begin{exampleBox} \begin{examples}}{ \end{examples} \end{exampleBox}}
\newenvironment{claim}{\begin{clmBox} \begin{clm}}{ \end{clm} \end{clmBox}}
\newenvironment{defn}{\begin{defBox} \begin{definition}}{ \end{definition} \end{defBox}}

