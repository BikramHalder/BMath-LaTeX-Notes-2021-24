\chapter*{Lecture 1, January 24}
\addcontentsline{toc}{chapter}{Lecture 1, January 24}
\setcounter{chapter}{1}

[Note: \qed \quad marks the end of a proof. If used immediately after a statement to be proved, indicates that the proof is trivial and left as an exercise to the reader]

\subsection*{Assumptions:}

\begin{itemize}
\item
  $\N$, the set of all natural numbers, is defined by $\N = \{0, 1, \dots \}$.

\item
  $\Z$, the set of all integers, is defined the usual way.
  
\item
  $\Z_+$, the set of all non negative integers, is defined by $\Z_+ = \{0, \pm 1, \pm 2, \dots \}$.

\item
  Any function $f \colon [a, b] \subset \R \to \R$ shall always be bounded.

\end{itemize}

\section{Partitions}

\begin{defn}
  A partition $P$ of $I = [a, b] \subset \R$ is a set of reals $\{x_0, x_1, \dots ,x_n \}$ for some $n \in \N$ such that
  \[
    x_0 < x_1 < \dots < x_n
  \]
  We shall denote the interval $[x_{j-1}, x_j]$ by the expression $I_j$.
\end{defn}

\begin{defn}
  If $I = (a, b)$ or $(a, b]$ or $[a, b)$ or $[a, b]$, we define
  \[
    \abs{I} = b - a
  \]
  We shall informally refer to $\abs{I}$ as the \textit{length} of $I$.
\end{defn}

\begin{claim}
  If $P = \{a = x_0, x_1, \dots ,x_n = b \}$ is a partion of $I = [a, b] \subset \R$,
  \[
    \abs{I} = \sum_{i=1}^n \abs{I_i}
  \]
  \hfill\qed
\end{claim}

\begin{claim}
  If $P$ and $\tilde{P}$ are both partitions of an interval $[a, b] \subset \R$, so is $P \cup \tilde{P}$. \qed
\end{claim}

\begin{defn}
  \label{jan24_def_Mm}
  \hfill
  \begin{enumerate}[label = \arabic*), itemsep = 10pt]
  \item
    We define $\pf[a, b]$ to be the set of all partitions (not just those of a fixed cardinality) of $[a, b]$. If the interval is clear from the context, we shall suppress it, writing $\pf[a, b]$ as $\pf$.

  \item
    Let $f \colon [a, b] \subset \R \to \R$ be a (bounded) function.
    Given a partition $P = \{a = x_0, x_1, \dots ,x_n = b \}$ of an interval $I = [a, b] \subset \R$, we define
    \[
      M_j = \sup_{x \in I_j} f(x)
      \qquad
      \text{and}
      \qquad
      m_j = \inf_{x \in I_j} f(x)
    \]
    for all $1 \leq j \leq n$. We also define
    \[
      M = \sup_{x \in I} f(x)
      \qquad
      \text{and}
      \qquad
      m = \inf_{x \in I} f(x)
    \]
    
  \end{enumerate}
\end{defn}

\begin{claim}
  If $S_1 \subset S_2 \subset \R$,
  \[
    \sup S_1 \leq \sup S_2
    \qquad
    \text{and}
    \qquad
    \inf S_1 \geq \inf S_2
  \]
  \hfill\qed
\end{claim}

\begin{corr}
  Using the notation of item 2 of definition \ref{jan24_def_Mm},
  \[
    m \leq m_j \leq M_j \leq M
  \]
  for all $1 \leq j \leq n$.
\end{corr}

\begin{proof}
  After choosing $S_1$ and $S_2$ to be the relevant images of $f$ (see definition \ref{jan24_def_Mm}), the statement follows trivially. \hfill
\end{proof}

\begin{defn}
  Given an interval $[a, b] \subset \R$, we define $\bb{B}[a, b]$ to be the set of all bounded functions from $[a, b]$ to $\R$.
\end{defn}
