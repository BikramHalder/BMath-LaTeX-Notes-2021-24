\chapter*{Lecture 16 \\ February 28, Part A}
\addcontentsline{toc}{chapter}{Lecture 16 \\ February 28, Part A}
\setcounter{chapter}{16}
\setcounter{section}{0}

\section{Error Analysis of the Newton Raphson method}

We begin by assuming that we are to use the Newton Raphson method to find a root $\alpha$ of a function $f$. We start with an initial guess of $\alpha$, which we denote by $x_0$.
The iteration used in the Newton Raphson method is
\begin{equation}
  \label{feb28:def:iter}
  x_{n+1} = x_n + \frac{f(x_n)}{f'(x_n)} \qquad \forall(n \geq 0)
\end{equation}
Note that if any of the $f'(x_n)$'s are 0, the iteration immediately fails at that stage. This is a flaw present in the Newton Raphson method that the bisection method doesn't suffer from.

In order to carry out the error analysis, we have to make some assumptions which we list below.
\begin{itemize}
\item
  There exists a $\rho > 0$ such that $f$ is continuously differentiable atleast twice in $[\alpha - \rho, \alpha + \rho]$. This assumption must be taken on faith.
\item
  $f'(\alpha) \neq 0$. If this assumption is not true, it is not possible for the iteration to converge to $\alpha$, as the term $\frac{f(x_n)}{f'(x_n)}$ would blow up to infinity if it did.
\end{itemize}

Note that the two assumptions listed above imply that $f' \neq 0$ in some neighbourhood of $\alpha$; by restricting our domain we can assume that $f' \neq 0$ in $[\alpha - \rho, \alpha + \rho]$.

Next, we assume that for some $n$ $x_n$ is sufficiently close to $\alpha$. Sufficiently close as in sufficiently close for the manipulations that follow. Then, using a Taylor Expansion and cutting it off after 3 terms, we can write
\begin{alignat*}{3}
  &&f(\alpha)
  &=
    f(x_n) + (\alpha - x_n)f'(x_n) + \frac 12 (\alpha - x_n)^2f''(x_n) \\
  \implies&&
  0
  &\lao{1}
  f(x_n) + (\alpha - x_n)f'(x_n) + \frac 12 (\alpha - x_n)^2f''(x_n) \\
  \implies&&
  0
  &\lao{2}
  \frac{f(x_n)}{f'(x_n)} + (\alpha - x_n) + (\alpha - x_n)^2\frac{f''(x_n)}{2f'(x_n)} \\
\end{alignat*}
where equality (1) follows from the fact that $\alpha$ is a root of $f$, equality (2) is obtained by dividing both sides of the equation by $f'(x_n)$, which we can do since $f'(x_n)$ is close to $f'(\alpha) \neq 0$, which in turns hold because $f'$ is continuous (by assumption). Using equation \ref{feb28:def:iter}, we obtain
\[
  \alpha - x_{n+1} = (\alpha - x_n)^2\frac{f''(x_n)}{2f'(x_n)}
\]
%%% Local Variables:
%%% mode: latex
%%% TeX-master: "../NC"
%%% End:
