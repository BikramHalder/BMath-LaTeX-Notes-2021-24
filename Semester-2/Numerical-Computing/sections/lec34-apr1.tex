\chapPreamble{34}{April 1, Part B}


\section{Quadrature}

We have a function $f$ about we don't know much and we want to calculate the integral 
$I = \int_a^b f(x) dx$, where  $a,b,f(x)$ are given. Out of the many ways of solving this 
problem, a series of them can be classified into a series of formulae called the 
\textbf{Newton-Cotes Formulae}, some of which are:
\begin{itemize}
    \item Trapezoidal Rule
    \item Simpson's Rule (Most popular for "smooth" functions $f$ and when  $b-a$ is finite.)
    \item Milne's Rule
\end{itemize}

\subsection{Simpson's Rule}

The defining feature of a Newton-Cotes formulation is that we have uniform partitioning 
$\Delta _u$ of  $b-a$ into (say)  $n$ sub-intervals. Let  $h = \frac{b-a}{n}$ be the length of
each sub-interval. Then our partition can be specified by the two numbers $(n,h)$. 
Let $x_0 = a$, $x_n = b$ and  $x_i = x_0 + ih,\ 0 \le  i \le  n$. 
Define $I_j = [x_{j-1}, x_j],\ j = 1,\ldots,n$ be the $jth$ sub-interval.
\\
Now we create an interpolating polynomial $P_n(x) \in \Pi _n$ such that  
$P(x_i) = f(x_i) = f_i,\ \forall i = 0,\ldots,n$. We shall use this polynomial to approximate
the integral.
Note that $x_k - x_i = (k-i)h$. So, we have the condition that $x_k = x_i \implies k = i$.

Recall that the \textbf{Lagrange Interpolation} gives us the polynomial 
$P_n(x) = \sum \limits _{i=0} ^n f_i L_i$ such that $L_i(x) = \prod \limits _{k=0 (k \neq i)} 
^ n$.
\\
Let $t = \frac{x-a}{h}$ for any $ x \in [a,b]$. Here, we shall scale/convert the variable $x$
using the dimensionless variable $t$. Using this, we get $x_i-x_k = h(i-k)$. So, we can write 
\begin{align*}
    L_i(x) &= \prod \limits _{k=0 (k \neq i)} ^n \frac{(x-x_k)}{(x_i - x_k)} \\
           &= \prod \limits _{k=0 (k \neq i)} ^n \frac{t-k}{i-k} = \Phi _i(t) 
.\end{align*}

Therefore, we can finally write 
\begin{align*}
    \int \limits _a ^b P_n(x) dx &= \sum \limits _{i=0} ^n f_i \int \limits _a^b L_i(x) dx \\
                                 &= \sum \limits  _{i=0} ^n h f_i \int \limits _{i=0}^n \Phi _i(t) dt \\
                                 &= h \sum \limits _{i=0}^n f_i \alpha_i
\end{align*}
where $\alpha _i = \int \limits _{t=0} ^n \Phi _i (t) dt$. Note that the weights $\alpha _i$ 
depend only on  $n$ and do depend on  $a,b$. 

\begin{example}
    Suppose we are given $n=2$, the support ordinates $f_0,f_1,f_2$ and limits $a=0, b=2$. 
    What is our approximation to  $I$?
    \\
    By Lagrange Interpolation, we first get $P_2(x) \in \Pi _2$. Then, using \textbf{Simpson's Rule}, We get
    \begin{align*}
        \int \limits _0^2 P_2(x) dx &= \frac{h}{3} [f_0 + 4f_1 +f_2]\\
                                    &= \frac{1}{3}[f_0 + 4f_1 +f_2] \  (h=1)
    .\end{align*}
\end{example}


\subsection{Trapezoidal Rule}

\subsection{Error}

The error is given by $\int \limits _a^b P_n(x)cdx - \int \limits _a^b f(x) dx$.

This error term for the \textbf{Trapezoidal Rule} is given by  $\frac{h^{3}}{12}f^{(2)}( \xi )$ 
for some $\xi \in [a,b]$. The same for \textbf{Simpson's Rule} is given by  
$\frac{h^5}{90}f^{(4)}f(\xi)$ for some $\xi \in [a,b]$.

Apply the rules in a composite form:- \\
\textbf{Trapezoidal Rule}\\
$\frac{h}{2[f(x_0) + f(x_1)]}$. We apply the Trapezoidal Rule to every interval 
$I_i = [x_i, x_{i+1}]$ for $i = 0, \ldots ,n-1$ to get $\mathscr{I} _i = \frac{h}{2} [f(x_i)
 + f(x_{i+1})]$. Therefore, the total integral becomes 
 \begin{align*}
     T(h) &= \sum \limits _{i=0} ^{n-1} \mathscr{I} _i \\
          &= h[\frac{f(a)}{2} + f(a+h) + f(a+2h) + \ldots + f(b-h) + \frac{f(b)}{2}]
 \end{align*}
 
 \section{Approximation (Composite) for Simpson's Rule}

 Trapezoidal Rule is an $O(2)$ method while Simpson's Rule is an $O(4)$ method.


%%% Local Variables:
%%% mode: latex
%%% TeX-master: "../NC"
%%% End:
