\chapter*{Lecture 18 \\ March 4}
\addcontentsline{toc}{chapter}{Lecture 18 (March 4)}
\setcounter{chapter}{18}
\setcounter{section}{0}

\section{Fixed point methods}

Instead of trying to find a root to an equation of the form $f(x) = 0$, we can try to find a root to an equation of the form $x - f(x) = x$. Defining $g(x) \coloneqq x - f(x)$, we see that the task of finding a root of $f$ is equivalent to finding a \textbf{fixed point} of $g$ (a fixed point of a function $g$ is a real $\alpha$ such that $g(\alpha) = \alpha$).

We shall deal mainly with non-linear functions, as finding the roots/fixed points of linear functions can be done by employing the techniques of linear algebra, which we have already covered.

Therefore, we now set for ourselves the task of finding fixed points $\alpha$ of a function $g$. But first, we recall some theorems that we proved in previous lectures

\begin{thm}
  \hfill
  
  If the following conditions obtain,
  \begin{itemize}
  \item
    $g \colon [a, b] \to [a, b]$.
    
  \item
    $g$ is continuous on $[a, b]$.
    
  \end{itemize}
  then $g$ has a (not necessarily unique) fixed point in $[a, b]$.
  \hfill\qed
\end{thm}

\begin{thm}
  \label{mar4:thm:ufp}
  \hfill
  
  If the following conditions obtain,
  \begin{itemize}
  \item
    $g \colon [a, b] \to [a, b]$.

  \item
    $g$ is continuous on $[a, b]$.

    \item
    $g'$ exists in $[a, b]$.

  \item
    $g'$ is continuous on $[a, b]$.

  \item
    $\lambda \coloneqq \max_{x \in [a, b]} \abs{g'(x)} < 1$.

  \end{itemize}
  then $g$ has a \textbf{unique} fixed point in $[a, b]$.
  \hfill\qed
\end{thm}
Note that we have the same problem here as when we did the bisection method; the problem of finding a suitable interval ($[a, b]$ in this case). But let us assume that you have found such an interval, and proceed.

Under the assumptions of theorem \ref{mar4:thm:ufp}, if we start with an initial guess $x_0 \in [a, b]$, and define subsequent guesses using the recursion
\begin{defn}
  \label{mar4:def:iter}
  \[
    x_{n+1} = g(x_n) \qquad \forall(n \geq 0)
  \]
\end{defn}
then if we denote the unique fixed point of $g$ in $[a, b]$ by $\alpha$, we have
\begin{thm}
  \[
    \lim_{n\to\infty} x_n = \alpha
  \]
\end{thm}

\begin{proof}
  Assuming the preconditions of theorem \ref{mar4:thm:ufp}, $g \colon [a, b] \to [a, b]$. Combining that with the fact that $x_0 \in [a, b]$, it is easy to see by induction that $x_n \in [a, b]$ for all $n \geq 0$.

  Now, by definition \ref{mar4:def:iter}, and the fact that $g(\alpha) = \alpha$, we have
  \[
    \alpha - x_{n+1} = g(\alpha) - g(x_n) = g'(c_n)(\alpha - x_n)
  \]
  for some $c_n \in (\alpha, x_n)$, by the \emph{mean value theorem}.
\end{proof}

[Stopped at 20:00 $+$ a few seconds]

%%% Local Variables:
%%% mode: latex
%%% TeX-master: "../NC"
%%% End:
