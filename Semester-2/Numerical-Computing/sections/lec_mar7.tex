\chapter*{Lecture 20\\March 7, Part B}
\addcontentsline{toc}{chapter}{Lecture 100, March 7}
\setcounter{chapter}{4}
\setcounter{section}{0}

\section{Introduction}

In numerical analysis, we are often provided with a table of values of a function f(x) with respect to some argument x. Examples might be log tables, tables for various statistical distributions or non-analytic integrals etc. This approach, although extremely convenient for practical purposes, has its limitations. The most obvious one being that a table can only contain finitely many values of a function, from which we need to estimate the intermediate values. This is where interpolation becomes invaluable.\\

Interpolation basically deals with approximating a function given its value at some support points. It is used, as the name suggests, to approximate $f(x^*)$, where $x*$ lies between two of the given support points.
