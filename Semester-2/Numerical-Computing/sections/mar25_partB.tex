\chapPreamble{30}{March 25, Part B}

\section{Uniqueness of Spline Function}

We now prove that under the hypothesis of \texty{Theorem} $\ref{thm2:mar25A}$, in each of the three cases there exists an unique spline function.

We restate the \texty{Theorem} $\ref{thm2:mar25A}$ as follows: 

\begin{thm}\label{thm1:mar25B}
    Let $ \Delta = \{ a = x_0 < x_1 < \cdots < x_n = b \} $
    be a partition of the interval $[a,b]$ and let $f \in \mathcal{K}^2[a,b]$. And suppose we are given the values $Y = (y_0, y_1, \dots, y_n)$ such that $f(x_i) = y_i, \ \forall \, i \in \{0,1,\dots,n\}$. Then
    \[
        0 \leq \| f - S_{\Delta}(Y; \cdot) \|^2 = \|f\|^2 - \| S_{\Delta}(Y; \cdot) \|^2    
    \]
    holds for every spline function $S_{\Delta}(Y; \cdot)$ provided one of the following conditions is true:
    \begin{enumerate}[label = (\alph*)]
        \item $S_{\Delta}^{(2)}(a) = S_{\Delta}^{(2)}(b) = 0 $. 
        
        \item $f \in \mathcal{K}_p^2[a,b]$ and $S_{\Delta}(Y; \cdot) \in \mathcal{K}^2_p[a,b]$.
        
        \item $S'_{\Delta}(Y;a) = f'(a)$ and $S'_{\Delta}(Y;b) = f'(b)$. 
    \end{enumerate}

    \vspace{0.1cm} 

    Furthermore in each of the cases the spline function $S_{\Delta}(Y; \cdot)$ is uniquely determined.
\end{thm}

\begin{prf}
    In the proof of \texty{Theorem} $\ref{thm2:mar25A}$, we already shown that the spline function minimizes the semi-norm if any of the conditions (a), (b) or (c) is met. So now we only need to check the uniqueness.

    Suppose there exists two spline functions $S_{\Delta}(Y; \cdot)$ and $\tilde{S}_{\Delta}(Y; \cdot)$ satisfying the hypothesis. But then since $S_{\Delta}(Y; \cdot), \tilde{S}_{\Delta}(Y; \cdot) \in \mathcal{K}^2[a,b]$, so we can use $\tilde{S}_{\Delta}(Y; \cdot)$ to play the role of $f \in \mathcal{K}^2[a,b]$. But then from the semi-norm minimizing property of spline function (\texty{Theorem} $\ref{thm2:mar25A}$), we get that 
    \begin{equation}\label{eq1:mar25B}
        \| \tilde{S}_{\Delta}(Y; \cdot) - S_{\Delta}(Y; \cdot) \|^2 = \| \tilde{S}_{\Delta}(Y; \cdot) \|^2 - \| S_{\Delta}(Y; \cdot) \|^2 \geq 0
    \end{equation} 
    But then we can switch the roles of $\tilde{S}_{\Delta}(Y; \cdot)$ and $S_{\Delta}(Y; \cdot)$, i.e., in this time we can take $S_{\Delta}(Y;\cdot)$ to play the role of the $f \in \mathcal{K}^2[a,b]$, then we get 
    \begin{align*}
        \| \tilde{S}_{\Delta}(Y; \cdot) - S_{\Delta}(Y; \cdot) \|^2 
        &= \| S_{\Delta}(Y; \cdot) - \tilde{S}_{\Delta}(Y; \cdot) \|^2 \\
        &= \| S_{\Delta}(Y; \cdot) \|^2 - \| \tilde{S}_{\Delta}(Y; \cdot) \|^2 \\ 
        &\overset{(\ref{eq1:mar25B})}{\leq} 0
    \end{align*}
    Thus we get 
    \[
        \| \tilde{S}_{\Delta}(Y; \cdot) - S_{\Delta}(Y; \cdot) \|^2 = 0 
    \]
    which gives us 
    \begin{equation}\label{eq2:mar25B}
        \int_a^b \left| \tilde{S}^{(2)}_{\Delta}(Y; \cdot) - S^{(2)}_{\Delta}(Y; \cdot) \right|^2\dd{x} = 0
    \end{equation}
    but then since $\tilde{S}_{\Delta}(Y;\cdot) \in \mathcal{C}^2[a,b]$ and $S_{\Delta}(Y; \cdot) \in \mathcal{C}^2[a,b]$, we get that equation $(\ref{eq2:mar25B})$ holds if and only if 
    \begin{equation}\label{eq3:mar25B}
        \tilde{S}_{\Delta}^{(2)}(Y; \cdot) \equiv S_{\Delta}^{(2)}(Y; \cdot)
    \end{equation}
    but then twice intergrating both sides of equation $(\ref{eq3:mar25B})$, we get that 
    \[
        \tilde{S}_{\Delta}(Y; \cdot) \equiv S_{\Delta}(Y; \cdot) + cx + d
    \]  
    but then we can easily compute $c$ and $d$ using the fact that $\tilde{S}_{\Delta}(Y;a) = S_{\Delta}(Y;a)$ and $\tilde{S}_{\Delta}(Y;b) = S_{\Delta}(Y;b)$. This gives us $c = d = 0$, and hence we have 
    \[
        \tilde{S}_{\Delta}(Y;\cdot) \equiv S_{\Delta}(Y;\cdot)  
    \] 
    which completes the proof of the uniqueness.
\end{prf}

\ 

We end this chapter, with some remarks on the three conditions (a), (b) and (c) in \texty{Theorem} $\ref{thm1:mar25B}$.

\begin{rmk}
    \begin{enumerate}[label = (\roman*)]
        \item The spline function arising from condition (a), is regarded as \texty{natural spline function}, as it does not have any other dependency on the function $f$, which we are trying to interpolate.
        \item On the other hand (b), is restricted to only functions $f \in \mathcal{K}^2_p[a,b]$.
        \item And finally, (c) kind of restricts the range of the function $f$, since we are fixing the values $f'(a) = y_0' $ and $f'(b) = y_n'$, where $y_0' = S'_{\Delta}(Y; x_0)$ and $y_n' = S'_{\Delta}(Y; x_n)$.
    \end{enumerate}
\end{rmk}

