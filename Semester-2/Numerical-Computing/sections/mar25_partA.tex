\chapPreamble{29}{March 25, Part A}

\section{Theoretical Foundations for Cubic Spline Interpolation}

Let $\Delta = \{a = x_0 < x_1< \cdot < x_n = b\}$ be a partition of the interval $[a,b]$. 

\begin{defn}\label{defn1:mar25A}
    A cubic spline function $S_{\Delta}$ on $\Delta$ is real function $S_{\Delta} : [a,b] \to \bb{R}$, with the properties:
    \begin{enumerate}
        \item $S_{\Delta} \in \mathcal{C}^2[a,b]$, i.e., $S_{\Delta}$ is twice continuously differentiable on $[a,b]$. 
        \item $S_{\Delta \mid [x_i, x_{i+1}]} \in \Pi_3$, i.e., coincides on every subinterval $[x_i, x_{i+1}]$ with a polynomial of degree at most three, for all $i = 0,1,\dots,n-1$. 
    \end{enumerate}
\end{defn}

We further define an \textit{interpolating spline function} as follows:
\begin{defn}\label{defn2:mar25A}
    For a finite sequence $Y := (y_0, y_1, \dots, y_n)$ of $n+1$ real numbers, we denote by 
    \[
        S_{\Delta}(Y; \cdot)  
    \]
    as an \textit{interpolating spline function} $S_{\Delta}$ with $S_{\Delta}(Y; x_i) = y_i$ for all $ i \in \{0,1,\dots,n\}$.
\end{defn}

and we have already discussed in \texty{Lecture 28}, that such a interpolating spline function is unique if any one of the three following condition is satisfied:

\begin{enumerate}[label = (\alph*)]
    \item
      $S''_\Delta(Y ; a) = S''_\Delta(Y ; b) = 0$.
    \item
      $S^{(k)}_\Delta(Y ; a) = S^{(k)}_\Delta(Y ; b)$ for $k = 0, 1, \text{ and } 2$.
    \item
      $S'_\Delta(Y ; a) = y'_0$, and $S'_\Delta(Y ; b) = y'_n$, where $y'_0$ and $y'_n$ are any fixed real numbers.
\end{enumerate}

Let $m \in \bb{N}$ and we shall now consider the following sets  
\begin{equation}\label{eq1:mar25A}
    \mathcal{K}^m[a,b] := \{ f : [a,b] \to \bb{R} \mid f^{(m-1)} \mbox{ is absolutely continuous on } [a,b] \mbox{ and } f^{(m)} \in \mathcal{L}^2[a,b] \}
\end{equation}
and we define 
\begin{equation}\label{eq2:mar25A}
    \mathcal{K}^m_p [a,b] := \{ f \in \mathcal{K}^m[a,b] \mid f^{(k)}(a) = f^{(k)}(b), \ \forall \, k = 0,1,\dots,m-1 \}
\end{equation}

So particularly for $m = 2$, we have $f \in \mathcal{K}^2_p[a,b]$, if 
\begin{itemize}
    \item $f^{(0)}$ and $f^{(1)}$ are absolutely continuous and $f^{(2)}$ exists. 
    \item $f^{(2)} \in \mathcal{L}^2[a,b]$, i.e., $\int_a^b |f^{(2)}(x)|^2 \dd{x}$ exists and is finite. 
    \item And finally $f^{(0)}(a) = f^{(0)}(b)$ and $f^{(1)}(a) = f^{(1)}(b)$.
\end{itemize}
note that the first two conditions are enough for $f \in \mathcal{K}^2[a,b]$.

\subsection{Holladay's Identity} 

Then note that $S_{\Delta} \in \mathcal{K}^3[a,b]$, and further if $S_{\Delta}^{(k)}(a) = S_{\Delta}^{(k)}(b)$ for $k = 0,1,2$ then $S_{\Delta} \in \mathcal{K}^3_p[a,b]$.

Also note that for all $ f \in \mathcal{K}^2[a,b]$, we can define the following \texty{semi-norm} 
\begin{equation}\label{eq3:mar25A}
    \| f \|^2 := \int_a^b |f(x)|^2 \dd{x} 
\end{equation} 

Now with all that definitions in mind, we are ready to define the following identity due to \texty{Holladay},
\begin{thm}\label{thm1:mar25A}\texty{(Holladay's Identity)}
    Let $f \in \mathcal{K}^2[a,b]$ and let \[\Delta = \{ a = x_0 < x_1 < \cdots < x_n = b \} \] be a partition of the interval $[a,b]$, and if $S_{\Delta}$ is a spline function with knots at $x_i \in \Delta$, then 
    \[
        \| f - S_{\Delta} \|^2 = \| f \|^2 - \| S_{\Delta} \|^2 - 2 \left[ (f'(x) - S'_{\Delta}(x))S^{(2)}_{\Delta}(x) \Big\vert_a^b - \sum_{i=1}^n (f(x) - S_{\Delta}(x)) S^{(3)}_{\Delta}(x) \Big\vert_{x_{i-1}^+}^{x_i^-}  \right]  
    \] 
\end{thm}

\begin{prf}
    We have 
    \begin{align*}
        \| f - S_{\Delta} \|^2 
        &= \int_a^b \left|f^{(2)}(x) - S^{(2)}_{\Delta}(x)\right|^2 \dd{x} \\ 
        &= \int_a^b \left|f^{(2)}(x)\right|^2 \dd{x} - 2 \int_a^b f^{(2)}(x)S_{\Delta}^{(2)}(x) \dd{x} + \int_a^b \left|S_{\Delta}^{(2)}(x)\right|^2 \dd{x} \\ 
        &= \int_a^b \left|f^{(2)}(x)\right|^2 \dd{x} - \int_a^b \left|S^{(2)}_{\Delta}(x)\right|^2 \dd{x} - 2 \int_a^b \left(f^{(2)}(x) - S_{\Delta}^{(2)}(x)\right) S_{\Delta}^{(2)}(x) \dd{x} \\ 
        &= \| f \|^2 - \| S_{\Delta} \|^2 - 2 \int_a^b \left(f^{(2)}(x) - S_{\Delta}^{(2)}(x)\right) S_{\Delta}^{(2)}(x) \dd{x}
    \end{align*}
    Now observe that using \texty{intergration by parts} we can write 
    \begin{align*}
        \int_{x_{i-1}}^{x_i} \left(f^{(2)}(x) - S_{\Delta}^{(2)}(x)\right) S_{\Delta}^{(2)}(x) \dd{x} &= \left(f^{(1)}(x) - S_{\Delta}^{(1)}(x)\right) S_{\Delta}^{(2)}(x) \Big\vert_{x_{i-1}}^{x_i} - \int_{x_{i-1}}^{x_i} \left(f^{(1)}(x) - S_{\Delta}^{(1)}(x)\right) S_{\Delta}^{(3)}(x) \dd{x} 
    \end{align*}
    which can be further broken down as 
    \begin{align*}
        \int_{x_{i-1}}^{x_i} \left(f^{(1)}(x) - S_{\Delta}^{(1)}(x)\right) S_{\Delta}^{(3)}(x) \dd{x} &= \left(f(x) - S_{\Delta}(x)\right) S_{\Delta}^{(3)}(x) \Big\vert_{x_{i-1}^+}^{x_i^-} - \int_{x_{i-1}}^{x_i} \left(f(x) - S_{\Delta}(x)\right) S_{\Delta}^{(4)}(x) \dd{x} \\ 
        &\overset{(1)}{=} \left(f(x) - S_{\Delta}(x)\right) S_{\Delta}^{(3)}(x) \Big\vert_{x_{i-1}^+}^{x_i^-} 
    \end{align*}
    where $(1)$, follows from the fact that $S_{\Delta \mid (x_{i-1},x_i)} \in \Pi_3$, and hence $S^{(4)}_{\Delta} \equiv 0 $ on $(x_{i-1},x_i)$, and combining the two results we got from \texty{intergration by parts} we get that 
    \begin{align*}
        \int_{a}^{b} \left(f^{(2)}(x) - S_{\Delta}^{(2)}(x)\right) S_{\Delta}^{(2)}(x) \dd{x} 
        &= \sum_{i=1}^n \int_{x_{i-1}}^{x_i} \left(f^{(2)}(x) - S_{\Delta}^{(2)}(x)\right) S_{\Delta}^{(2)}(x) \dd{x} \\ 
        &= \sum_{i=1}^n \left(f(x) - S_{\Delta}(x)\right) S_{\Delta}^{(3)}(x) \Big\vert_{x_{i-1}}^{x_i} - \sum_{i=1}^n \left(f(x) - S_{\Delta}(x)\right) S_{\Delta}^{(3)}(x) \Big\vert_{x_{i-1}^+}^{x_i^-} \\ 
        &= (f(x) - S_{\Delta}(x)) S^{(3)}_{\Delta}(x) \Big\vert_a^b - \sum_{i=1}^n \left(f(x) - S_{\Delta}(x)\right) S_{\Delta}^{(3)}(x) \Big\vert_{x_{i-1}^+}^{x_i^-}
    \end{align*}
    which completes the proof of \texty{Theorem} $\ref{thm1:mar25A}$.
\end{prf}

\subsection{Minimum Semi-norm Property of Spline Functions}

\begin{thm}\label{thm2:mar25A}
    Let $ \Delta = \{ a = x_0 < x_1 < \cdots < x_n = b \} $
    be a partition of the interval $[a,b]$ and let $f \in \mathcal{K}^2[a,b]$. And suppose we are given the values $Y = (y_0, y_1, \dots, y_n)$ such that $f(x_i) = y_i, \ \forall \, i \in \{0,1,\dots,n\}$. Then
    \[
        0 \leq \| f - S_{\Delta}(Y; \cdot) \|^2 = \|f\|^2 - \| S_{\Delta}(Y; \cdot) \|^2    
    \]
    holds for every spline function $S_{\Delta}(Y; \cdot)$ provided one of the following conditions is true:
    \begin{enumerate}[label = (\alph*)]
        \item $S_{\Delta}^{(2)}(a) = S_{\Delta}^{(2)}(b) = 0 $.
        
        \item $f \in \mathcal{K}_p^2[a,b]$ and $S_{\Delta}(Y; \cdot) \in \mathcal{K}^2_p[a,b]$.
        
        \item $S'_{\Delta}(Y;a) = f'(a)$ and $S'_{\Delta}(Y;b) = f'(b)$. 
    \end{enumerate}
\end{thm}

\begin{prf}
    First note that from the definition of spline interpolation function $S_{\Delta}(Y; \cdot)$, we have 
    \[
        \lim_{x \to x_k^-} f(x) = f(x_k) = y_k = \lim_{x \to x_k^-} S_{\Delta}(Y;x)  
    \]
    and 
    \[
        \lim_{x \to x_k^+} f(x) = f(x_k) = y_k = \lim_{x \to x_k^+} S_{\Delta}(Y;x)
    \]
    for any $k \in \{0,1,\dots,n\}$. Thus the term 
    \[
        \sum_{i=1}^n \left(f(x) - S_{\Delta}(Y;x)\right) S_{\Delta}^{(3)}(Y;x) \Big\vert_{x_{i-1}^+}^{x_i^-} = 0  
    \]
    and hence using \texty{Holladay's identity} (\texty{Theorem} $\ref{thm1:mar25A}$), we get 
    \begin{equation}\label{eq4:mar25A}
        \| f - S_{\Delta}(Y;\cdot) \|^2 = \|f \|^2 - \| S_{\Delta}(Y;\cdot) \|^2 - 2 \left( f'(x) - S'_{\Delta}(Y;x)\right)S_{\Delta}^{(2)}(Y;x) \Big\vert_a^b
    \end{equation}
    But then its easy to observe that for any of the three conditions (a), (b) or (c), the term 
    \[
        \left( f'(x) - S'_{\Delta}(Y;x)\right)S_{\Delta}^{(2)}(Y;x) \Big\vert_a^b = 0
    \]
    Thus we have 
    \begin{equation}\label{eq5:mar25A}
        \| f - S_{\Delta}(Y; \cdot) \|^2 = \| f \|^2 - \| S_{\Delta}(Y; \cdot) \|^2
    \end{equation}
    but it trivially follows from the definition of \texty{semi-norm}, that $\| f - S_{\Delta}(Y;)\| \geq 0$, which completes our proof. 
\end{prf}

But why is the spline Interpolation, so important? We end this chapter with the following geometric interpretation of the spline interpolation. 

\subsection{Geometric Interpretation of Cubic Spline Functions}

\begin{defn}\label{defn3:mar25A}
    Let $f \in \mathcal{K}^2[a,b]$, then the curvature of $f$ at the point $x \in [a,b]$, denoted by $\kappa_x$ is defined by 
    \[
        \kappa_x = \frac{f^{(2)}(x)}{(1+f'(x)^2)^{\frac{3}{2}}}  
    \]
\end{defn}

Now suppose we have $|f'(x)| \ll 1$, i.e., $|f'(x)|$ is very small compared to $1$, then in that case we have 
\[
    \kappa_x \approx f^{(2)}(x)  
\]
But observe that the term 
\[ 
    \|f\|^2 = \int_a^b |f^{(2)}(x)|^2 \dd{x}    
\]
is kind of measure for the total curvature of the function $f$ in the interval $[a,b]$, but from \texty{Theorem} $\ref{thm2:mar25A}$, we have already shown that if the spline function satisfies certain conditions, then it minimizes the term $\|f\|^2$. So we can say that a spline function is the \textit{"smoothest"} function to interpolate the given support points $(x_i,y_i)$.