\chapter*{Lecture 26 \\ March 18}
\addcontentsline{toc}{chapter}{Lecture 26 (March 18)}
\setcounter{chapter}{26}
\setcounter{section}{0}

\section{Algorithms for Hermite interpolation}
Unfortunately Neville's algorithm and Newton's divided difference algorithm cannot be used to carry out Hermite interpolation, as they assume the $x$-coordinates of the support points are all different. Our task here then, is to develop algorithms that \textit{can} be used to carry out Hermite interpolation. We begin by introducing a technical tool that'll help us with that task.

\subsection{Virtual Abscissae}
Suppose you are given $m$ distinct reals $x_0 < x_1 < \dots < x_m$, and for all $0 \leq i \leq m$, at real $x_i$ you are given the values $f^{(0)}(x_i), f^{(1)}(x_i), \dots ,f^{(n_i - 1)}(x_i)$, for some $n_i \in \Z$. Those reals and values are the support points we shall work with. Define
\[
  n = \sum_{i = 0}^m n_i
\]
Now list the support points as follows:
\begin{enumerate}
\item
  First, list $f^{(0)}(x_0), f^{(1)}(x_0), \dots ,f^{(n_0 - 1)}(x_0)$
\item
  Then, append the list $f^{(0)}(x_1), f^{(1)}(x_1), \dots ,f^{(n_1 - 1)}(x_1)$ in that order to the pre-existing list.
\item
  Then, append the list $f^{(0)}(x_2), f^{(1)}(x_2), \dots ,f^{(n_2 - 1)}(x_2)$ in that order to the pre-existing list.
\item
  Continue this way until you...
\item[.]
\item[.]
\item[.]
\item[m.]
  Finally, append the list $f^{(0)}(x_m), f^{(1)}(x_m), \dots ,f^{(n_m - 1)}(x_m)$ in that order to the pre-existing list.
\end{enumerate}
Since there are $n$ support points in total, there must be $n$ items in our final list. Now define
\begin{defn}
  
\end{defn}

%%% Local Variables:
%%% mode: latex
%%% TeX-master: "../NC"
%%% End:
