\chapPreamble{38}{April 11, Part B}

\section{Basic Properties of Orthogonal Polynomials}
Now that we have shown that there exists polynomials $p_j \in \bar{\Pi}_j$, for $j = 0,1,2,\dots$, such that 
\[
    (p_i,p_j) = 0, \quad \mbox{ for } i \neq j
\]   

Thus we have $\{p_1, \dots, p_k\}$ is an orthonormal basis for the inner product space $\Pi_k$, with inner product defined by the \textit{weight functions} 
\[
    (f,g) := \int_a^b w(x) f(x) g(x) \dd{x}, \quad f,g \in \Pi_j
\]  
Thus any polynomial $p \in \Pi_k$ is clearly representable as a linear combination of the orthogonal polynomials $p_i$, $i \leq k$, which in fact gives us the following lemma: 

\begin{lem}\label{lem1:apr11B}
    $(p,p_n) = 0$, for all $p \in \Pi_{n-1}$.
\end{lem}
\begin{prf}
    We have $\{p_0,\dots,p_{n-1}\}$ is an orthonormal basis for $\Pi_{n-1}$, and hence there exists scalars $a_0, a_1, \dots, a_{n-1} \in \bb{R}$ such that 
    \[
        p(x) = \sum_{i=0}^{n-1} a_i p_i(x)  
    \]
    and thus by linearity of inner product we get 
    \begin{align*}
        (p,p_n) &= \left( \sum_{i=0}^{n-1} a_i p_i, p \right) \\ 
        &= \sum_{i=0}^{n-1} a_i (p_i , p) \\ 
        &= 0
    \end{align*}
\end{prf}

\begin{thm}\label{thm1:apr11B}
    The roots of the polynomial $p_n$ are real and simple.
\end{thm}

\begin{prf}
    We the consider the roots of $p_n$ which are of odd multiplicities, i.e., we consider the roots where $p_n$ changes sign. WLOG let them be $x_1, \dots, x_l$, where $l \leq n$, and we define the polynomial $q(x)$ by 
    \begin{equation}\label{eq1:apr11B}
        q(x) = \prod_{j=1}^l (x-x_j) \in \bar{\Pi}_l
    \end{equation}
    But then observe that the polynomial $p_n(x)q(x)$ does not change sign (as all the roots are of even multiplicities now). Thus we must have 
    \[
        (p_n,q) = \int_a^b w(x) p_n(x) q(x) \dd{x} \neq 0  
    \]
    but then $\mbox{deg}(q) = l$ cannot be strictly less than $n$, as otherwise we would have $(p_n, q) = 0$ (from \texty{Lemma} $\ref{lem1:apr11B}$), thus it must be the case that $l = n$. But then all the roots of $p_n$ have multiplicity $1$, and thus we get that roots of $p_n$ are real and simple.
\end{prf}

\ 

We need this lecture by giving a glimpse of what we will do in the next lecture. 

\section{Glimpse of Exact Integration using Gaussian Integration}

Consider $x_1, x_2, \dots, x_n$ be the roots of the orthogonal polynomial $p_n$. We consider the following matrix: 

\begin{equation}\label{eq2:apr11B}
    A := \begin{bmatrix}
        p_0(x_1) & p_0(x_2) & \cdots & p_0(x_n) \\ 
        p_1(x_1) & p_1(x_2) & \cdots & p_1(x_n) \\ 
        \vdots & \vdots & \ddots & \vdots \\ 
        p_{n-1}(x_1) & p_{n-1}(x_1) & \cdots & p_{n-1}(x_n) 
    \end{bmatrix}
\end{equation}
we will show that the following matrix is in fact nonsingular, thus we can consider the system of linear equations 
\begin{equation}\label{eq3:apr11B}
    \sum_{i=1}^n p_k(x_i) w_i = \begin{cases}
        (p_0,p_0) = \int_a^b w(x) \dd{x} & \mbox{ if } k = 0 \\ 
        0 & \mbox{ if } k = 1, \dots, n-1
    \end{cases}
\end{equation}
then as we will see, we get $w_i > 0$ for $ i = 1,\dots,n$ and in fact we have 
\begin{equation}\label{eq4:apr11B}
    \int_a^b w(x)p(x) \dd{x} = \sum_{i=1}^n w_i p(x_i)
\end{equation}
holds for all polynomial $p \in \Pi_{2n-1}$.
