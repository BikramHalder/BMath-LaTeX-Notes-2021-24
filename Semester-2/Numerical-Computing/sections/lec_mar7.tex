\chapter*{Lecture 20\\March 7, Part A}
\addcontentsline{toc}{chapter}{Lecture 20, March 7, Part A}
\setcounter{chapter}{20}


\setcounter{section}{0}

\section{Interpolation}

In numerical analysis, we are often provided with a table of values of a function $f(x)$ with respect to some argument $x$. Examples might be log tables, tables for various statistical distributions, non-analytic integrals etc. This approach, although extremely convenient for practical purposes, has its limitations. The most obvious one being that a table can only contain finitely many values of a function, from which we need to estimate the intermediate values. This is one of the places where interpolation becomes invaluable. It can also be used to develop polynomial approximations of transcendental or even non-analytic functions.\\
Interpolation basically deals with approximating a function given its value at some support points. It is used, as the name suggests, to approximate $f(x^*)$, where $x^*$ lies between two of the given support points. This gives rise to an "approximate function" which can be used to get a reasonable guess of what $f(x^*)$ should be. The function formed must satisfy all the support points.\\
The subject of interpolation has also given rise to one of the most impactful algorithms of the $20^{th}$ century, the \textbf{Fast Fourier Transform}.

\section{Lagrange's Interpolation Formula}

Let $\Pi_n$ be the set if all polynomials with degree $\leq n$. Thus, any $P(x) \in \Pi_n$ can be written in the form:
$$P(x) = \sum_{j = 0}^n a_j x^j\;\;\;\;\;a_j \in \mathbb{R}\;\;\forall j= 1(1)n$$
Given (n+1) support points , the Lagrange interpolation gives a polynomial of degree n which satistfies all the points. So, if $(x_i,f_i)$ for $i = 0(1)n$ be the support points (where $f_i$ are the values of the function at $x_i$), we will have:
$$P(x_i) = f_i    \;\;\; \forall i = 0(1)n$$

\begin{rmk}
	The support points must have distinct abscissae. Otherwise, one or more support points are either redundant or contradictory. So, we need:
	$$x_i \neq x_j \;\;\;\forall\;i\neq j$$
	Throughout this lecture, we will assume that this conditions is met.
\end{rmk}

We will show that given these conditions, $\exists! P \in \Pi_n$ such that the interpolation condition is staisfied, i.e, $P(x_i) = f_i\;\;\;\forall i=0(1)n$. We first show the existence by construction of \textbf{Lagrange Polynomials} and then uniqueness.\\

\begin{defn}
	Given (n+1) support points $(x_i,f_i)$, $i = 0(1)n$ with distinct abscissae, we define the $i^{th}$ Lagrange Polynomials as follows:
	$$L_i(x) = \prod^n_{\substack{ j = 0\\j \neq i}} \frac{x - x_j}{x_i - x_j}$$
\end{defn}


Then, taking $P(x) =  \sum_{i=0}^{n} f_i L_i(x)$, it is easy to see that
$$L_i(x_k) = \delta_{ik} =
	\begin{cases}
		1, & \text{if }i=k    \\
		0, & \text{otherwise}
	\end{cases}
$$

$$\therefore P(x_k) = \sum_{i=0}^{n} f_i L_i(x_k) =
	\sum_{i=0}^{n} f_i \delta_{ik} = f_k$$

i.e, it satisfies the interpolation condition.
This proves the \textbf{existence} of $P$.$\hfill \blacksquare$\\[.2 cm]
To prove the uniqueness, we assume two polynomials $P_1(x),P_2(x) \in \Pi_n$ which satisfy the interpolation conditions. Then the polynomial:
$$P_3(x) = P_1(x)-P_2(x) \in \Pi_n$$ has at least $n+1$ roots $\{x_i: i = 0(1)n\}$, and is thus identically zero. Hence, $P_1(x) \equiv P_2(x)$, and the interpolating polynomial is \textbf{unique}.$ \hfill\blacksquare $\\[0.2 cm]
This, in turn, shows that the Lagrange Polynomials are the only ones satisfying $L_i(x_k) = \delta_{ik}$, as it is a special case of the Lagrange Interpolation.

\begin{rmk}
	As the polynomial is unique, we have different approaches for finding $f(x^*)$:
	\begin{enumerate}
		\item Finding the coefficients of $P(x)$ and then calculating $P(x^*)$.
		\item Finding the value of $P(x^*)$ directly from the support points.
	\end{enumerate}
	The two interpolation methods discussed in this lecture, Lagrange Interpolation and Neville Interpolation, are best suited to approach (2).\\
	Approach (1) is more convenient when the interpolation needs to be done multiple times. This is best done using Newton's Divided difference method, as will be discussed later.
\end{rmk}

\begin{example}
	Say the following support points have been provided. Find the interpolated value at $x^* = 2$.\\[0.2 cm]
	\begin{center}
		\begin{tabular}{|r||r|r|r|}
			\hline
			$x_i$ & 0 & 1 & 3 \\
			\hline
			$f_i$ & 1 & 3 & 2 \\
			\hline
		\end{tabular}
	\end{center}
\end{example}
\begin{soln}
	We simply compute the values of the Individual Lagrange Polynomials at $x^*$ and then put them together.
	$$L_0(x^* = 2) = \frac{(x^* - x_1)(x^* - x_2)}{(x_0 - x_1)(x_0 - x_2)} = -\frac{1}{3}$$
	Similarly,
	$$L_1(2) = 1 \text{  and  } L_2(2) = \frac{1}{3}$$
	Thus $$P(x^*) = \sum_{i = 0}^2 f_i L_i(x^*) = -\frac{1}{3}+ 3 + \frac{2}{3} = \frac{10}{3}$$
\end{soln}


