\chapter*{Lecture 5, Febuary 4}
\addcontentsline{toc}{chapter}{Lecture 5, Febuary 4}
\setcounter{chapter}{5}
\setcounter{section}{0}

Let $f \in \mathcal{B}[a,b]$. We have already shown that for any partition $P \in \mathcal{P}[a,b]$, we have 
\begin{equation}\label{eq:LSU}
    L(f,P) \leq S(f,P) \leq U(f,P) 
\end{equation}

\section{Riemann Sums as a tool for computing integrals}

\begin{defn}
    Given $f \in \mathcal{B}[a,b]$, we say that 
    \begin{equation}\label{eq1:feb4}
        \lim_{\|P\|\to 0} S(f,P) = \lambda 
    \end{equation}
    for some $\lambda \in \bb{R}$, if for every $\eps > 0$, there exists a $\delta > 0$ such that 
    \[ |S(f,P) - \lambda| < \eps \]
    for all $P \in \mathcal{P}[a,b]$, and tag set $T_P$, such that $\|P\| < \delta$.
\end{defn}

Observe that whenever the limit in equation $(\ref{eq1:feb4})$ exists, the limit is unique, and hence our definition does not have any ambiguity. 

\begin{thm}\label{thm1:feb4}
    Let $f \in \mathcal{B}[a,b]$. Then $f \in \mathcal{R}[a,b]$, if and only if there exists a $\lambda \in \bb{R}$ such that \[ \lim_{\|P\| \to 0} S(f,P) = \lambda\]
    in this case we have $\int_a^b f = \lambda$.
\end{thm}
\begin{prf}
    Let $f \in \bb{R}[a,b]$, and suppose $\lambda = \int_a^b f$. Fix $\eps > 0$. Then from \texty{Darboux Criterion} we get that there exists a $\delta > 0$ such that 
    \[ U(f,P) - L(f,P) < \eps \]
    for all $P \in \mathcal{P}[a,b]$ such that $\|P\| < \delta$. Then note that 
    \begin{align*}
        L(f,P) > U(f,P) - \eps \geq \overline{\int_a^b}f - \eps = \lambda - \eps \\ 
        U(f,P) < L(f,P) + \eps \geq \underline{\int_a^b}f + \eps = \lambda + \eps
    \end{align*}
    Then from equation $(\ref{eq:LSU})$ we get that 
    \[ \lambda - \eps < L(f,P) \leq S(f,P) \leq U(f,P) < \lambda + \eps \]
    Hence, we get that $|S(f,P) - \lambda| < \eps$ for all $P \in \mathcal{P}[a,b]$ and tag set $T_P$, such that $\|P\| < \delta$.

    Conversely, suppose \[ \lim_{\|P\| \to 0} S(f,P) = \lambda \]
    Then for $\eps > 0$, we get there there exists a $\delta > 0$ such that $|S(f,P) - \lambda| < \displaystyle{\frac{\eps}{2}}$ for all $P \in \mathcal{P}[a,b]$ and tag set $T_P$, such that $\|P\| < \delta$. Suppose $P = \{x_0, \dots, x_n\}$, and let $M_j = \sup_{x \in I_j} f(x)$ where $I_j = [x_{j-1},x_j]$ for $j = 1, \dots, n$. Then note that from properties of supremum of a set we get that $\forall \ \eps > 0$, there exists a $\zeta_j \in I_j$ such that \[ M_j - \frac{\eps}{2(b-a)} < f(\zeta_j) \leq M_j \] So if we choose our tag set to be the set of all these $\zeta_j$'s, we get that 
    \begin{align*}
        S(f,P) &= \sum_{j=1}^n f(\zeta_j) \Delta x_j \\ 
               &> \sum_{j=1}^n (M_j - \frac{\eps}{2(b-a)}) \Delta x_j \\ 
               &= \sum_{j=1}^n M_j \Delta x_j - \frac{\eps}{2(b-a)} \sum_{j=1}^n \Delta x_j \\ 
               &= U(f,P) - \frac{\eps}{2}
    \end{align*}
    But now since 
    \[
        \lambda-\frac{\eps}{2} < S(f,P) < \lambda + \frac{\eps}{2}  
    \]
    for all $P \in \mathcal{P}[a,b]$ and tag set $T_P$, such that $\|P\| < \delta$, we get that 
    \[
        U(f,P) - \frac{\eps}{2} < S(f,P) < \lambda + \frac{\eps}{2} \Rightarrow U(f,P) < \lambda + \eps   
    \]
    and 
    \[
        \lambda - \frac{\eps}{2} < S(f,P) \leq U(f,P) \Rightarrow U(f,P) > \lambda - \eps   
    \]
    hence, combining everything we get that 
    \begin{equation}\label{eq3:feb4}
        |U(f,P) - \lambda| < \eps, \ \forall \, P \in \mathcal{P}[a,b] \mbox{ such that } \|P\| < \delta
    \end{equation}
    Now using similar arguments its easy to show that 
    \begin{equation}\label{eq4:feb4}
        |L(f,P) - \lambda| < \eps, \ \forall \, P \in \mathcal{P}[a,b] \mbox{ such that } \|P\| < \delta
    \end{equation}
    Thus from equation $(\ref{eq3:feb4})$ and $(\ref{eq4:feb4})$ we get that 
    \[
        U(f,P) - L(f,P) \leq |U(f,P) - \lambda| + |L(f,P)-\lambda| < 2\eps  
    \] 
    for all $P \in \mathcal{P}[a,b]$ such that $\|P\| < \delta$, hence from \texty{Darboux's criterion} it follows that $f \in \mathcal{R}[a,b]$, and finally since 
    \[ \lambda - \eps < L(f,P) \leq \int_a^b f \leq U(f,P) < \lambda + \eps \]
    It follows that $\int_a^b f = \lambda$, which completes the proof.
\end{prf}

\ 

\texty{Theorem} $\ref{thm1:feb4}$ is a nice tool for computing integrals, if we already know that $f \in \mathcal{R}[a,b]$.

\begin{thm}\label{thm2:feb4}
    Suppose $f \in \mathcal{R}[a,b]$ and $\{P_n\}_{n\in\bb{N}} \subset \mathcal{P}[a,b]$ such that $\| P_n \| \to 0$ as $n \to \infty$. Then 
    \[ 
        \lim_{n \to \infty} S(f,P_n) = \int_a^b f    
    \]
\end{thm}
\begin{prf}
    Since $f \in \mathcal{R}[a,b]$, using \texty{Darboux criterion} we get that for all $\eps > 0$, there exists a $\delta > 0$, such that 
    \[
        U(f,P) - L(f,P) < \eps, \ \forall \, P \in \mathcal{P}[a,b] \mbox{ such that } \|P\| < \delta  
    \]
    Now since $\|P_n\| \to 0$, thus for all $\delta > 0$, there exists a $N \in \bb{N}$ such that $\|P_n\| < \delta, \ \forall \, n > N$. Thus we get 
    \[ 
        U(f, P_n) - L(f, P_n) < \eps, \ \forall \, n > N    
    \]
    But then we get that 
    \[ 
        \left( U(f,P_n) - \int_a^b f \right) + \left( \int_a^b f - L(f,P_n) \right) < \eps, \ \forall \, n > N    
    \]
    Note that $U(f,P_n) - \int_a^b f \geq 0$ and $\int_a^b f - L(f,P_n) \geq 0$, hence we have 
    \begin{align}\label{eq5:feb4}
        &&0 \leq U(f,P_n)-\int_a^b f < \eps &&\mbox{and} &&0 \leq \int_a^b f - L(f,P_n) < \eps &&\forall \, n > N
    \end{align}
    Thus from equation $(\ref{eq5:feb4})$, we conclude that 
    \[
        \lim_{n\to\infty} L(f,P_n) = \lim_{n\to\infty} U(f,P_n) = \int_a^b f  
    \]
    Finally from \texty{Squeeze theorem} we get that 
    \[
        \int_a^b f = \lim_{n\to\infty} L(f,P_n) \leq \lim_{n\to\infty}S(f,P_n) \leq \lim_{n\to\infty} U(f,P_n) = \int_a^b f  
    \]
    and hence $\lim_{n\to\infty} S(f,P_n) = \int_a^b f$, which completes the proof.
\end{prf}

\begin{exampleBox}
    \texty{Note:} \textit{The limit $\lim_{n\to\infty} S(f,P_n) = \int_a^b f$ does not depend on the tag set, so we can choose our favourite tag set for computing the integral, given that we know that the function is Riemann integrable.}
\end{exampleBox}

Suppose we know that $f \in \mathcal{R}[a,b]$, then consider the following partition \[ a = x_0 < x_1 = x_0 + \frac{1}{n}(b-a) < x_2 = x_0 + \frac{2}{n}(b-a) < \cdots < x_n = x_0 + (b-a) = b \]
Then note that if $P = \{x_0,\dots,x_n\}$, then $\|P\| = \frac{b-a}{n}$, thus we get that $\|P\| \to 0$ as $n \to \infty$. And since $f \in \mathcal{R}[a,b]$ we know that 
\begin{align*}
    \lim_{n\to\infty} S(f,P_n) = \int_a^b f
\end{align*}
Now we choose the tag set to be the points $\{x_0,x_1,\dots,x_{n-1}\}$, then we get that 
\begin{equation}\label{eq6:feb4}
    \lim_{n\to\infty}\left(\frac{b-a}{n}\sum_{j=1}^n f \left(a+\frac{j-1}{n}(b-a) \right)\right) = \int_a^b f
\end{equation}
Thus, if we know that the function is Riemann integrable then the integral can be written as a limit of \texty{Newton sums}.

\section{What else can we say about the set $\mathcal{R}[a,b]$ ?}

Observe that the set of bounded functions on $[a,b]$, i.e., $\mathcal{B}[a,b]$ forms a vector space over the field of real numbers, with addition on $\mathcal{B}[a,b]$ as sum of functions, and scalar multiplication as product of a function with a real number. 
\begin{itemize}
    \item Associativity and commutativity follows trivially.
    \item Additive identiy is the zero function and obviously we have $\mathbf{0} \in \mathcal{B}[a,b]$, and multiplicative identiy is $1$.
    \item And if $f \in \mathcal{B}[a,b]$, then since $|f| = |-f|$, we get that $-f \in \mathcal{B}[a,b]$, and since $f + (-f) = \mathbf{0}$, we get that additive inverse exists for all $f \in \mathcal{B}[a,b]$.
    \item Distributive properties hold trivially.
\end{itemize} 
Also its not difficult to show that if $f : [a.b] \to [c,d]$ and $g : [c,d] \to \bb{R}$ such that both $f,g \in \mathcal{B}[a,b]$, then $g \circ f : [a,b] \to \bb{R} \in \mathcal{B}[a,b]$.

The next question that arises immediately is can we say $\mathcal{R}[a,b]$ is a vector space over the field of real numbers? Well, with similar operations as in case of $\mathcal{B}[a,b]$, its again not very difficult to show that indeed $\mathcal{R}[a,b]$ is a vector space over the field of real numbers. Now consider the function $\mathcal{I} : \mathcal{R}[a,b] \to \bb{R}$ given by 
\begin{equation}\label{eq7:feb4}
    \mathcal{I}(f) = \int_a^b f, \ \ \forall \, f \in \mathcal{R}[a,b]
\end{equation}

\begin{thm}
    Suppose $\mathcal{I} : \mathcal{R}[a,b] \to \bb{R}$ be defined as in equation $(\ref{eq7:feb4})$, then following conditions are true 
    \begin{enumerate}
        \item[(i)] For all $r,s \in \bb{R}$ and $f,g \in \mathcal{R}[a,b]$ we have 
        \[
            \mathcal{I}(rf+sg) = r\mathcal{I}(f) + s\mathcal{I}(g)
        \]  
        i.e., $\mathcal{I}$ is a \texty{linear functional}.
        
        \item[(ii)] $\mathcal{I}$ preserves the order, i.e., if $f, g \in \mathcal{R}[a,b]$ and we have $f(x) \leq g(x), \ \forall \, x \in [a,b]$, then \[ \mathcal{I}(f) \leq \mathcal{I}(g) \]
        
        \item[(iii)] Let $c \in (a,b)$ then we have 
        \[
            \mathcal{I}(f) = \int_a^c f + \int_c^b f  
        \]  
    \end{enumerate}
\end{thm}