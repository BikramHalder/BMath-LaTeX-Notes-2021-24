\chapter*{Lecture 21\\March 7, Part B}
\addcontentsline{toc}{chapter}{Lecture 21, March 7, Part B}
\setcounter{chapter}{21}

\section{Neville Interpolation}
This is a more systematic way of implementing Lagrange Interpolation. Here, we make polynomials which satisfy a subset of the support points, and use those to construct the final polynomial in a step by step manner. As mentioned previously, we mainly use this method to find the interpolated value and not the coefficients of the polynomial.\\

\begin{defn}
	For a given set of support points $(x_i,f_i)$,$i = 0(1)n$ with distinct abscissae, we define Neville polynomials of order k as
	$$P_{i_0i_1\dots i_k} \in \Pi_k \text{    s.t.    } P_{i_0i_1\dots i_k}(x_i) = f_i\;\;\forall i \in \{i_0,i_1,\dots,i_k\}$$
	where $i_\alpha$, $\alpha = 0(1)n $ as distinct elements of $\{0,1,2,\dots,n\}$
\end{defn}

We note that all these polynomials are unique, as shown in the previous section.\\
Thus, $P_{i}(x) = f_{i}\;\;\forall i = 0(1)n$. These constant polynomials form the zeroth level  Neville Polynomials.\\
The key idea in Neville's Algorithm is a method to find $P_{i_0i_1\dots i_k}(x)$ from $P_{i_0i_1\dots i_{k-1}}(x)$ and $P_{i_1i_2\dots i_k}(x)$. Thus, knowing the zeroth level $P_0, P_1,\dots, P_n$, we can find the first level $P_{0,1}, P_{1,2},\dots, P_{n-1,n}$ and so on. Once we reach the $n^{th}$ level, we find the polynomial $P_{0,1\dots, n}$ which satisfies:
$$P_{0,1\dots, n} \in \Pi_n \text{    s.t.    } P_{0,1\dots, n}(x_i) = f_i\;\;\forall i = 0(1)n $$
This is precisely the interpolating polynomial $P(x)$ we are after!\\[0.2 cm]
Neville's Algorithm could be summarised in the following table.\\[0.2 cm]


\begin{center}
	\begin{tabular}{|ccccc|}
		\hline
		$k = 0$              & 1           & \dots   & n-1                 & n                 \\
		\hline
		$ P_0 = f_0$         & $P_{0,1}$   & $\dots$ & $P_{0,1,\dots,n-1}$ & $P_{0,1,\dots n}$ \\
		$ P_1 = f_1$         & $P_{1,2}$   & $\dots$ & $P_{1,2,\dots,n}$   &                   \\
		$\vdots$             & $\vdots$    &         &                     &                   \\
		$ P_{n-1} = f_{n-1}$ & $P_{n-1,n}$ &         &                     &                   \\
		$ P_{n} = f_{n}$     &             &         &                     &                   \\
		\hline
	\end{tabular}
\end{center}



\begin{props}
	The corresponding recurrence relation for Neville Interpolation is:
	$$P_{i_0i_1\dots i_k}(x) = \frac{(x-x_{i_0})P_{i_1i_2\dots i_k}(x)-(x-x_{i_k})P_{i_0i_1\dots i_{k-1}}(x)}{x_{i_k} - x_{i_0}}$$
\end{props}
\begin{prf}
	Let $$G(x):=\frac{(x-x_{i_0})P_{i_1i_2\dots i_k}(x)-(x-x_{i_k})P_{i_0i_1\dots i_{k-1}}(x)}{x_{i_k} - x_{i_0}}$$
	As $P_{i_1i_2\dots i_k}\text{, }P_{i_0i_1\dots i_{k-1}}(x) \in \Pi_{k-1}$, we get $G(x) \in \Pi_k$.
	Now, $$G(x_{i_0}) = \frac{-(x_{i_0}-x_{i_k})P_{i_0i_1\dots i_{k-1}}(x_{i_0})}{x_{i_k} - x_{i_0}} =  P_{i_0i_1\dots i_{k-1}}(x_{i_0}) = f_{i_0}$$
	Similarly, we have $G(x_{i_k}) = f_{i_k}$.\\
	For $\alpha = 1(1)k-1$, we have
	$$P_{i_1i_2\dots i_k}(x_{i_\alpha}) = P_{i_0i_1\dots i_{k-1}}(x_{i_\alpha}) = f_{i_\alpha}$$
	$$\implies G(x_{i_\alpha}) = \frac{(x-x_{i_0})f_{i_\alpha}-(x-x_{i_k})f_{i_\alpha}}{x_{i_k} - x_{i_0}} = f_{i_\alpha}\;\;\;\forall \alpha = 1(1)k-1$$
	So, $G(x)$ satisfies all the conditions of $P_{i_0i_1\dots i_k}(x)$.\\
	Noting that $P_{i_0i_1\dots i_k}(x)$ is unique (from previous section), this completes the proof.
\end{prf}
\begin{rmk}
	In theory, Neville Interpolation is just a reformulation of Lagrange Interpolation. However, it cuts down the computations required significantly and is easier to implement. For instance, the computation of the next level only requires knowledge of the previous level.
\end{rmk}

\begin{example}
	We try to solve the same problem as before using Neville Interpolation. Given data is
	\begin{center}
		\begin{tabular}{|r||r|r|r|}
			\hline
			$x_i$ & 0 & 1 & 3 \\
			\hline
			$f_i$ & 1 & 3 & 2 \\
			\hline
		\end{tabular}
	\end{center}
	We need to find the interpolated value at $x^* = 2$.
\end{example}

\begin{soln}
	To begin with, we have\\
	$P_0(2) = 1 \text{, } P_1(2) = 3 \text{, } P_2(2) = 2$\\
	From this, we have
	\begin{equation}
		\begin{split}
			P_{0,1}(x^*=2) &= \frac{(x^*-x_0)P_1(x^*)-(x^*-x_1)P_0(x^*)}{x_1 - x_0} \\
			&= \frac{(2-0)P_1(2)-(2-1)P_0(2)}{1 - 0} = 5
		\end{split}
	\end{equation}
	Similarly, $P_{1,2}(2) = \frac{5}{2}$ and thus,
	$$P_{0,1,2}(x^* = 2) = \frac{(x^*-x_0)P_2(x^*)-(x^*-x_2)P_0(x^*)}{x_2 - x_0} = \frac{10}{3}$$
	As expected, this gives us the same answer as before.

	\begin{center}
		\begin{tabular}{|ccc|}
			\hline
			$k = 0$ & 1              & 2              \\
			\hline
			1       & 5              & $\frac{10}{3}$ \\
			3       & $-\frac{5}{2}$ &                \\
			2       &                &                \\

			\hline
		\end{tabular}
	\end{center}
\end{soln}