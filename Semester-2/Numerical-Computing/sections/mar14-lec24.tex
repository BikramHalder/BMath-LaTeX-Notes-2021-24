\chapPreamble{24}{March 14, Part A}

We know that the interpolating polynomial $P_{i_0i_1\ldotsi_k}$ is uniquely determined by its support points. So, it is invariant of any permutation of the indices $ i_0,i_1,\ldots,i_k$. Therefore we have the following result:
\begin{thm}
    The coefficients/divided-differences $f_{i_0i_1\ldotsi_k}$ are invariant of any permutations of the indices $ i_0,i_1,\ldots,i_k$.
\end{thm}

\begin{example}\label{ex1:lec24}

We take up an old example and form the \textit{divided-differences scheme} for the same:
    \begin{center}
    \begin{tabular}{c|ccc}
        $k$ &  $0$ &  $1$ &  $2$\\
        \hline
        $x_0=0$ & $f_0=1$ & \\
                & & $f_{01}=2$ & \\
        $ x_1=1$ & $ f_1=3$ & & $f_{012} = \frac{-5}{6}$\\
                 & &  $ f_{12}=\frac{-1}{2}$ &\\
        $ x_2=3$ & $ f_2=2$ & \\
    \end{tabular}
    \end{center}

Here $f_{01} = \frac{f_1-f_0}{x_1-x_0} = 2$, $f_{12} = \frac{f_2-f_1}{x_2-x_1}$ and $f_{012} = \frac{f_{12}-f_{01}}{x_2-x_0} = \frac{-5}{6}$.
 
So, we get our final interpolated polynomial as $P_{012}(x) = 1+2.(x-0)-\frac{5}{6}(x-0)(x-1)$.
\end{example}

\begin{rmk}\label{rmk1:lec24}
    We will always want to choose such a permutation of the indices that satisfies
    \[
        |\xi - x_{i_k}| \ge  |\xi - x_{i_{k-1}}|, \forall k=1,2,\ldots,n
    .\] 
which dampens the error during the evaluation of the Horner's Scheme.
\end{rmk}

In accordance of the remark (\ref{rmk1:lec24}), we have an algorithm to choose the preferred sequence of indices. For this, we assume that the support abscissae $x_i$ are in, say ascending, order. Then for each  $k>0$, we can choose the index  $i_k$ so that either  $i_k = min\{i_l|0\le l<k\}-1$ or $i_k - max\{i_l|0\le l<k\}+1$. So, instead of the descending-diagonal sequence of indices that we get from the divided-difference scheme, here we get a zig-zag path.

\begin{example}
    Continuing from Example \ref{ex1:lec24}, we can use the sequence $ i_0=1, i_1=2, i_3 = 0$ for interpolating the polynomial at $\xi = 2$. Then the corresponding Newton Representation is:
    \[
        P_{120}(x) = 3 - \frac{1}{2}(x-1) - \frac{5}{6} (x-1)(x-3)
    .\] 

\end{example}

\section{An alternate notation}

