\chapter*{Lecture 1, January 24}
\addcontentsline{toc}{chapter}{Lecture 1, January 24}
\setcounter{chapter}{1}
\setcounter{section}{0}

[Note: $\blacksquare$ marks the end of a proof, i.e., QED. If used immediately after a statement to be proved, indicates that the proof is trivial and left as an exercise to the reader]

\subsection*{Assumptions:}

\begin{itemize}
\item
  $\N$, the set of all natural numbers, is defined by $\N = \{0, 1, \dots \}$.

\item
  $\Z$, the set of all integers, is defined the usual way.
  
\item
  $\Z_+$, the set of all non-negative integers, is defined by $\Z_+ = \{0, \pm 1, \pm 2, \dots \}$.

\item
  Any function $f \colon [a, b] \subset \R \to \R$ shall always be bounded.

\end{itemize}

\section{Partitions}

\begin{defn}
  A partition $P$ of $I = [a, b] \subset \R$ is a set of reals $\{x_0, x_1, \dots ,x_n \}$ for some $n \in \N$ such that
  \[
    x_0 < x_1 < \dots < x_n
  \]
  We shall denote the interval $[x_{j-1}, x_j]$ by the expression $I_j$.
\end{defn}

\begin{defn}
  If $I = (a, b)$ or $(a, b]$ or $[a, b)$ or $[a, b]$, we define
  \[
    \abs{I} = b - a
  \]
  We shall informally refer to $\abs{I}$ as the \textit{length} of $I$.
\end{defn}

\begin{claim}
  If $P = \{a = x_0, x_1, \dots ,x_n = b \}$ is a partition of $I = [a, b] \subset \R$,
  \[
    \abs{I} = \sum_{i=1}^n \abs{I_i}
  \]
  \hfill\qed
\end{claim}

\begin{claim}
  If $P$ and $\tilde{P}$ are both partitions of an interval $[a, b] \subset \R$, so is $P \cup \tilde{P}$. \qed
\end{claim}

\begin{defn}
  \label{jan24:def:Mm}
  \hfill
  \begin{enumerate}[label = \arabic*), itemsep = 10pt]
  \item
    We define $\mathcal{P}[a, b]$ to be the set of all partitions (not just those of a fixed cardinality) of $[a, b]$. If the interval is clear from the context, we shall suppress it, writing $\mP[a, b]$ as $\mP$.

  \item
    Let $f \colon [a, b] \subset \R \to \R$ be a (bounded) function.
    Given a partition $P = \{a = x_0, x_1, \dots ,x_n = b \}$ of an interval $I = [a, b] \subset \R$, we define
    \[
      M_j = \sup_{x \in I_j} f(x)
      \qquad
      \text{and}
      \qquad
      m_j = \inf_{x \in I_j} f(x)
    \]
    for all $1 \leq j \leq n$. We also define
    \[
      M = \sup_{x \in I} f(x)
      \qquad
      \text{and}
      \qquad
      m = \inf_{x \in I} f(x)
    \]
    
  \end{enumerate}
\end{defn}

\begin{claim}
  If $S_1 \subset S_2 \subset \R$,
  \[
    \sup S_1 \leq \sup S_2
    \qquad
    \text{and}
    \qquad
    \inf S_1 \geq \inf S_2
  \]
  \hfill\qed
\end{claim}

\begin{corr}
  \label{jan24:corr:mMIneq}
  Using the notation of item 2 of definition \ref{jan24:def:Mm},
  \[
    m \leq m_j \leq M_j \leq M
  \]
  for all $1 \leq j \leq n$.
\end{corr}

\begin{proof}
  After choosing $S_1$ and $S_2$ to be the relevant images of $f$ (see definition \ref{jan24:def:Mm}), the statement follows trivially. \hfill
\end{proof}

\begin{defn}
  Given an interval $[a, b] \subset \R$, we define $\mathcal{B}[a, b]$ to be the set of all bounded functions from $[a, b]$ to $\R$.
\end{defn}

\begin{defn}
  \label{jan24:def:UL}
  Let $f \in \mB[a, b]$ and let $P = \{a = x_0, x_1, \dots , x_n = b \} \in \mP[a, b]$. Then, using the notation of item 2 of definition \ref{jan24:def:Mm}, the \textbf{upper Riemann sum} of $f$ with respect to $P$ is defined as
  \[
    U(f; P) = \sum_{i=1}^n M_i\abs{I_i}
  \]
  and the \textbf{lower Riemann sum} of $f$ with respect to $P$ is defined as
  \[
    L(f; P) = \sum_{i=1}^n m_i\abs{I_i}
  \]
\end{defn}

\begin{claim}
  Both $U(f; P)$ and $L(f; P)$ must exist.
\end{claim}

\begin{proof}
  The maxima and minima exist because $f \in\mB[a, b]$, and the sums exist because $P$ has finitely many elements (also called nodes).
  \hfill
\end{proof}

\begin{thm}
  Using the notation of definitions \ref{jan24:def:UL} and \ref{jan24:def:Mm}, given a function $f \in \mB[a, b]$,
  \[
    m(b-a) \leq L(f; P) \leq U(f; P) \leq M(b-a)
  \]
  for all $P \in \mP[a, b]$.
\end{thm}

\begin{proof}
  Let $P = \{a = x_0, x_1, \dots , x_n = b \}$. For all $1 \leq i \leq n$, using corollary \ref{jan24:corr:mMIneq}
  \[
    m\abs{I_i} \leq m_i\abs{I_i} \leq M_i\abs{I_i} \leq M\abs{I_i}
  \]
  Summing up over all $1 \leq i \leq n$, we obtain the desired inequality.
\end{proof}

\begin{corr}
  \label{jan24:corr:allBnd}
  For all $P \in \mP$, $L(f; P)$ and $U(f; P)$ are bounded by $m(b-a)$ and $M(b-a)$. \qed
\end{corr}

\begin{defn}
  Suppose that $f \in \mB[a, b]$. We define the \textbf{lower Riemann integral} of $f$ on $[a, b]$ to be
  \[
    \underline{\int_a^b} f = \sup\{L(f, P) \colon P \in \mP[a, b] \}
  \]
  and \textbf{upper Riemann integral} of $f$ on $[a, b]$ to be
  \[
    \overline{\int_a^b} f = \inf\{U(f, P) \colon P \in \mP[a, b] \}
  \]
  Note that both integrals must exist, since they are defined as the supremums/infimums of sets that by corollary \ref{jan24:corr:allBnd} are bounded, and because the reals are complete.
\end{defn}

\begin{defn}
  Suppose that $f \in \mB[a, b]$. We say that $f$ is \textbf{Riemann integrable} on $[a, b]$ if
  \[
    \underline{\int_a^b} f = \overline{\int_a^b} f
  \]
  We also define the \textbf{Riemann integral} of $f$ on $[a, b]$ to be
  \[
    \int_a^b f \coloneqq \underline{\int_a^b} f = \overline{\int_a^b} f
  \]
\end{defn}

The remainder of the notes covering the material done in Lecture 1 overlaps with the notes covering the material done in Lecture 2, so there is no point in reproducing it here.

